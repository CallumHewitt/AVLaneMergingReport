\chapter{Literature Notes}
\label{cha:Literature Notes}

\section{\citetitle{Dresner2004}}
\label{sec:Dresner2004}
\begin{itemize}
\item Introduction
\begin{itemize}
\item Motivated by traffic light problems -> My work is motivated by lane change problems evidenced in \citet{Laval2006}
\item Set in a world of fully autonomous vehicles.
\item Overpass is optimal solution, what is optimal for my work?
\end{itemize}
\item The Model
\begin{itemize}
\item Using a simplified model of real-world intersection traffic -> No turning, roughly same speed. Worth considering my model to start with. How can that model be adapted?
\item How do we measure success? 1) Safety is critical. No collisions allowed! 2) Efficiency
\item Throughput: How much traffic can be handled. Difficult to measure, qualitative claims only made.
\item Delay: Effect on overall journey of the vehicle. No vehicle's travel time sacrificed for another dramatically. Consider both average delay and maximum delay!
\end{itemize}
\item Overpass, Traffic Light Theory
\begin{itemize}
\item Simplifications due to car interactions and acceleration to calculate lower bound traffic light delays.
\end{itemize}
\item The Simulator
\begin{itemize}
\item Useful sizing stats in this section.
\item Spawning characteristics, driver properties, three actions the driver can take, relevant decision logic.
\item Testing simulator with no big changes. Constructing current system
\end{itemize}
\begin{itemize}
\item Simplification from earlier no longer applies to light model.
\item "Call ahead" system -> Could be applied to a centralised model for lane changing.
\item Intersection divided into reservation tiles -> could be applied to lane changing too.
\end{itemize}
\item Empirical Results
\begin{itemize}
\item Measuring overloaded systems vs light traffic
\item Increasing granularity tests -> At least as high as the number of lanes
\end{itemize}
\end{itemize}

\section{\citetitle{Laval2006}}
\label{sec:Laval2006}

\begin{itemize}
\item Introduction
\begin{itemize}
\item Attempt to create a qualitative understanding of lane changing impacts on traffic flow.
\item Lane change triggers disruption -> triggers other changes
\item Considers freeway as a series of interacting streams linked by lane changes
\item Combination of multiple stream models
\end{itemize}
\item The Model
\begin{itemize}
\item Based on the Kinematic Wave model: \url{https://en.wikipedia.org/wiki/Kinematic_wave}
\item Not really sure of the maths here. More research required.
\end{itemize}
\end{itemize}

\section{\citetitle{Kesting2007}}
\label{sec:Kesting2007}

\begin{itemize}
\item Introduction
\begin{itemize}
\item Drivers want to increase their own utility
\item Drivers have a strategic view of lane changes -> They have a target destination that might require them to change to a specific lane
\item "Politeness factor" - Drivers often consider the loss of utility of other drivers. Introducing a politeness parameter varies a driver's response from altruistic to egotistical
\item Optimal politeness parameter -> MOBIL: Minimizing Overall Braking Induced by Lane Changes
\item Consider US driving rules and European driving rules (Symmetric vs Asymmetric with reversed Asymmetric for the UK)
\end{itemize}
\item The lane-changing model MOBIL
\begin{itemize}
\item Safety criterion says that after a lane change deceleration of car behind doesn't exceed a given safety limit.
\item Incentive to change needs to be greater than the switching threshold, which is drivers utility gain + politeness factor * follower's utility gain
\item Right lane bias with left lane priority for European rule roads
\end{itemize}
\item Application to multi-lane traffic simulations
\begin{itemize}
\item Intelligent Driver Model -> Guarantee's crash free driving. \url{http://www.traffic-simulation.de/}
\item Maximum politeness = Maximum throughput on both lanes. As density of traffic starts to get over 20/km/lane lane changes decrease as fewer suitable gaps start to appear.
\end{itemize}
\end{itemize}

\section{\citetitle{Gipps1986}}
\label{sec:Gipps1986}
\begin{itemize}
\item Introduction
\begin{itemize}
\item Modelling individual driver behaviour makes it easier to deal with bottlenecks such as road works or accidents. This behaviour is easier to simulate at a driver level. Modelling only driver behaviour instead of systems for dealing with the overall system (centralised).
\item Drivers need to reconcile short and long term aims.
\item This paper refers to how the decision to change lanes is made, as opposed to the mechanics of changing lanes.
\end{itemize}
\item Concerning Driver Behaviour
\begin{itemize}
\item Three questions:
\begin{enumerate}
\item Is it possible to change lanes?
\item Is it necessary to change lanes?
\item Is it desirable to change lanes?
\end{enumerate}
\item Assumptions include: Drivers has a goal to travel from X to Y in safety within a given time. This is translated to a number of specific quantitative objectives.
\item Factors influencing driver changing lanes decision
\begin{enumerate}
\item Physically possible and safe to change lanes
\item Location of permanent obstructions
\item The presence of transit lanes
\item Driver's intended turning movement
\item Presence of heavy vehicles
\item Speed
\end{enumerate}
\end{itemize}
\item The Model
\begin{itemize}
\item The model covers the entire motorway experience. Entering -> Travelling -> Exiting
\item Behaviour changes based on proximity to driver's exit.
\item Fits in with car following model. Same as that adapted in \citet{Kesting2007}
\item Flowchart summarising decision process as well as mathematical decision process in paper.
\end{itemize}
\end{itemize}
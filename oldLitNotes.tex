OLD LIT NOTES

KAMALI
Kamali developed a model for an automated platoon, defining procedures for vehicles joining and leaving. 

A joining vehicle can integrate at either the back or the middle of the platoon. The vehicle first sends a join request to the platoon leader. If the vehicle is at the back of the platoon the leader sends an agreement and the vehicle follows its predecessor. If the vehicle requests to join in front of another platoon vehicle, the leader first asks the platoon vehicle to increase space; once the space is large enough for the joining vehicle, the leader sends an agreement. The joining vehicle then manoeuvres into the space and follows the preceding vehicle. Having now joined the platoon, the vehicle sends a confirmation to the leader. The leader then requests that the vehicle that gave way for the joining vehicle decreases their spacing back to normal.

A leaving vehicle sends a request to the leader. When it receives permission to leave the vehicle increases its spacing from its predecessor; once the vehicle is at its maximum distance from its predecessor the vehicle can change lanes. Once out of the convoy the vehicle sends an acknowledgement to the leader.

This model isn't very strict, acting as more of a set of requirements than a true model. The paper sets the requirements using pre-defined gaps, and has no strict calculations guiding following characteristics. It could be implemented using spacing rules from both the IDM and Gipps' model, however, by using V2V communication, the lead vehicle can control the actions of all vehicles in its platoon. Instead of using IDM or Gipps' model, the lead vehicle can control the gaps between vehicles so that they all increase and decrease simultaneously. The gaps could be based on the platoon's velocity, perhaps using \eqref{IDMSpacingFunction} from the IDM. By centralising control in this way, vehicle platoons can avoid the traffic shock effect \citep{Daganzo1994}.

ATAGOZIYEV CENTRALISED LANE MERGE
A centralised system for lane changing was described in a paper by Atagoziyev et al. in 2016 \citep{Atagoziyev2016}. This system uses roadside infrastructure to help groups of vehicles change lanes before they reach a 'critical-position', such as a motorway exit or intersection. The vehicles send their position and velocity information to the roadside infrastructure; the system then sends a number of orders to the vehicles such that they safely rearrange themselves into the correct lanes.  Because the distance travelled by the vehicles involved can be large, particularly at high speeds, the system would struggle to use a reservation tile based system as AIM did, instead Atagoziyev's system manages the gaps between vehicles to safely relocate vehicles into the correct position. A more comprehensive overview is given in \ref{subsec:Lane Changing to hit a target lane}.

Atagoziyev's system would only need to be applied during the approach to critical-positions. Vehicles could be managed using platoons or another vehicle following model until that point. A centralised system provides a single communication point which manages all of the vehicles that want to change lanes. This helps to reduce the volume of communications required and creates an entity with a global view of the vehicles' positions and goals. A V2V solution would most likely require more communications and may never obtain a complete picture of the situation, possibly leading to sub-optimal lane changing orders.

Note that Atagoziyev's system could be adapted, such that all of the vehicles communicate with the platoon leader instead of roadside infrastructure. Though this could be called a V2V communication solution, the effective solution is still considered centralised, as all decisions are made by one entity.

GIPPS AND MOBIL DECENT
Cost also becomes a major issue for centralised systems when you consider fast moving situations such as lane changing on a motorway. To implement Atagoziyev's model, vehicles must remain in range of the roadside infrastructure. This would mean that the infrastructure will have to continue on for a long distance, which could become very expensive, especially given the number of critical-positions on a motorway. Decentralised solutions reduce these costs massively.

Two examples of decentralised lane changing models are Gipps' 1986 driver decision model \citep{Gipps1986} and the MOBIL model developed by Kesting et al. in 2007 \citep{Kesting2007}. These models are decentralised and as such do not have to rely on roadside infrastructure in order to change lanes. This greatly reduces the cost of both implementations and allows the vehicles to be more flexible as to when they change lanes, no longer having to wait until they reach the lead up to a critical-position supported by roadside infrastructure. This flexibility means that vehicles could change lanes to increase their average velocity rather than just changing lanes in order to make a turn or leave the motorway at a critical-position. It also allows vehicles to deal with unexpected situations far from any roadside infrastructure. For example, a broken down car blocking a lane can be evaded. There is more information on Gipps' 1986 model and MOBIL in \ref{sec:Making lane changing decisions}.
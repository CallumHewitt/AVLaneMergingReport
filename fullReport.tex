\documentclass{UoYCSproject}
\title{Approaches to Autonomous Vehicle Merge Management}
\author{Callum Hewitt}
\supervisor{Lilian Blot}
\date{2nd May 2017}
\wordcount{11566}
\excludes{equations or appendices}
\BEng
\abstract{
In anticipation of a fully autonomous vehicle future, this project aims to develop and analyse systems that deal with lane merging. An existing autonomous vehicle simulator, used for vehicle intersections, was adapted and extended to develop merge simulations. The system itself proved unexpectedly difficult to work with, and further research will require the development of a more universal simulator. A merge management system based on a queue protocol was developed to manage incoming vehicles on a single lane to single lane merge. This was compared with an adaptation of the intersection management system. It was found that the intersection management system induced less delay on the vehicles than the queue system, particularly at high traffic levels. The queue protocol was tested further under different conditions. The angle at which the two lanes meet was found to have a substantial effect on the performance of the protocol. When the two lanes met at shallow angles, the queue system performed very poorly, but the performance improved rapidly as the angle approached 90\degree. The queue system was less effective at reducing delays on a lane when it had a higher speed limit than the lane it was merging with.The distance simulated before the vehicle reached the merge point was found to have no effect on the performance of the protocol when the distance was greater than 150\si{m}. It was concluded that an intersection protocol based system could perform better than the queue protocol if fully developed. The performance of the protocol at different angles, speed limits and simulated distances, provided insight into issues that merge systems will have to overcome in order to be integrated into real world infrastructure.
}
\usepackage{listings}
\usepackage{graphicx}
\usepackage[table]{xcolor}
\usepackage{usebib}
\usepackage{tikz}
\usepackage{amsmath}
\usepackage{amssymb}
\usepackage{gensymb}
\usepackage{color,soul}
\usepackage{multirow}
\usepackage{longtable}
\usepackage{appendix}
\usepackage{amssymb}
\usepackage{pifont}
\usepackage{siunitx}
\usepackage{float}
\usepackage{rotating}
\usepackage[section]{placeins}

\bibinput{bibliography}

\newcommand\todo[1]{\emph{{\textcolor{red}{TODO: #1}}}}
\newcommand\newtext[1]{\emph{{\textcolor{blue}{#1}}}}
\newcommand\revisit[1]{\hl{#1}}
\newcommand\citetitle[1]{``\usebibentry{#1}{title}''}
\newcommand{\cmark}{\ding{51}}
\newcommand{\xmark}{\ding{55}}

\graphicspath{{images/}}

\begin{document}
\maketitle
\listoffigures
\listoftables

\chapter{Introduction}
\label{cha:Introduction}

The terms 'autonomous vehicle' and 'self-driving car' were once thought of as science fiction, but as of recent, they have become our reality. Google's Self-Driving Car Project is gaining traction, with cars currently driving in Milton Keynes and four different US states \citep{GoogleCars}. Tesla Motors have deployed a beta version of their Autopilot system into all of their vehicles produced since September 2014. The system has been blamed for both saving and ending lives \citep{TeslaHospital} \citep{TeslaUnderInvestigation}. 2016 has been a big year for autonomous vehicles and with that comes an even bigger push for robust and secure autonomous systems.
The possible benefits of autonomous vehicles cover a lot of different areas of concern. 

The main issue it addresses is safety. Autonomous vehicles would be able to react to incidents on the road much more quickly than a human driver would. A human's 'thinking distance' can often determine whether someone survives an accident or not. This distance can also be greatly increased if the driver of the vehicles is under the influence of alcohol or narcotics. An autonomous vehicle however, would be able to react to accidents much more quickly than a human, reducing the thinking distance greatly, improving road safety.

\begin{figure}[htb]
\includegraphics[width=\textwidth]{stoppingDistances.jpg}
\caption{Diagram from Rule 126 in the UK Highway Code \citep{StoppingDistances}}
\end{figure}

Autonomous vehicles could also make transport more efficient. Research by Mersky in April 2016 suggested that fuel conservation control strategies could make autonomous vehicles up to 10\% more fuel efficient than current EPA fuel economy test results \citep{Mersky2016} \todo{Read this article}. Having vehicles which are fuel efficient is becoming increasingly important, with landmark climate change deals such as 'The Paris Agreement' introducing limits on greenhouse gas emissions globally. The introduction of electric vehicles into the car market is also an important factor to consider, as the range of such vehicles still has not managed to match that of their gasoline counterparts. More efficient driving strategies introduced by autonomous vehicles could reduce this gap.

Congestion contributes to fuel loss in quite a large way. In the US in 2014 an estimated 3.1 billion gallons (11.7 billion litres) of fuel was wasted due to congestion \citep{Schrank2015}. Automating typical driving activities and communications between vehicles, in situations such as lane changes, could reduce congestion and improve efficiency. Unsafe lane changes don't even have to result in a crash to cause delays. If a car brakes due to a car merging unsafely it can cause a ripple effect, creating a traffic jam.

Autonomous vehicles also offer a level of comfort not currently available today. In a world where autonomous vehicles are commonplace, it is not hard to imagine people doing work, reading or relaxing in their car instead of having to focus on driving. 

However, today there are still a number of concerns surrounding autonomous vehicles. One of the major concerns is over the reliability of the systems governing the vehicle. These systems need to be responsive and accurate and they cannot afford to fail in such safety critical environments. Already concerns over Tesla's Autopilot system are impacting the image of the company, and the system isn't even out of beta testing yet \citep{TeslaCriticised}. 

In order to address these concerns safely, we can create simulations which test our autonomous systems. These simulations can test the reliability of our systems. Researchers at the University of Texas set up the Autonomous Intersection Management (AIM) project, which aims to " create a scalable, safe, and efficient multiagent framework for managing autonomous vehicles at intersections" \citep{AIMProject}. The project managed to apply their tested intersection software in a mixed reality test using a real life autonomous vehicle \citep{Quinlan2010}, demonstrating how simulations are vital tools when testing these safety critical systems.

In this project we make a number of assumptions. Firstly we assume that the sensors resolving the positions of the vehicle and it's surrounding obstacles are perfectly accurate. We also assume that the vehicle can communicate reliably with other vehicles. These assumptions are existing areas of research for autonomous vehicles but are not considered in this paper. The main focus here is on how autonomous vehicles can self-organise to minimise delays in traffic with effective, safe lane merging.

The aims of this project are as follows:
\begin{itemize}
\item Attempt to generalise the AIM codebase such that other simulations can be created for non-intersection related situations.
\begin{itemize}
\item If the codebase proves difficult to refactor, new simulator code will need to be created
\end{itemize}
\item Use the new codebase to create a decentralised system for managing lane merging.
\item Use the new codebase to create a centralised system for managing lane merging.
\item Compare the effectiveness of both strategies.
\end{itemize}

Creating these simulations helps to determine the effectiveness of two different strategies and also provides a codebase within which future simulations for other situations can be created.
\chapter{Literature Review}
\label{cha:Literature Review}

\section{\citetitle{Dresner2004}}
\label{sec:Dresner2004}

\begin{itemize}
\item \newtext{Introduction}
\begin{itemize}
\item \newtext{Motivated by traffic light problems -> My work is motivated by lane change problems evidenced in \citet{Laval2006}}
\item \newtext{Set in a world of fully autonomous vehicles.}
\item \newtext{Overpass is optimal solution, what is optimal for my work?}
\end{itemize}
\item \newtext{The Model}
\begin{itemize}
\item \newtext{Using a simplified model of real-world intersection traffic -> No turning, roughly same speed. Worth considering my model to start with. How can that model be adapted?}
\item \newtext{How do we measure success? 1) Safety is critical. No collisions allowed! 2) Efficiency}
\item \newtext{Throughput: How much traffic can be handled. Difficult to measure, qualitative claims only made.}
\item \newtext{Delay: Effect on overall journey of the vehicle. No vehicle's travel time sacrificed for another dramatically. Consider both average delay and maximum delay!}
\end{itemize}
\item \newtext{Overpass, Traffic Light Theory}
\begin{itemize}
\item \newtext{Simplifications due to car interactions and acceleration to calculate lower bound traffic light delays.}
\end{itemize}
\item \newtext{The Simulator}
\begin{itemize}
\item \newtext{Useful sizing stats in this section.}
\item \newtext{Spawning characteristics, driver properties, three actions the driver can take, relevant decision logic.}
\item \newtext{Testing simulator with no big changes. Constructing current system}
\end{itemize}
\begin{itemize}
\item \newtext{Simplification from earlier no longer applies to light model.}
\item \newtext{"Call ahead" system -> Could be applied to a centralised model for lane changing.}
\item \newtext{Intersection divided into reservation tiles -> could be applied to lane changing too.}
\end{itemize}
\item \newtext{Empirical Results}
\begin{itemize}
\item \newtext{Measuring overloaded systems vs light traffic}
\item \newtext{Increasing granularity tests -> At least as high as the number of lanes}
\end{itemize}
\end{itemize}

\section{\citetitle{Laval2006}}
\label{sec:Laval2006}

\begin{itemize}
\item \newtext{Introduction}
\begin{itemize}
\item \newtext{Attempt to create a qualitative understanding of lane changing impacts on traffic flow.}
\item \newtext{Lane change triggers disruption -> triggers other changes}
\item \newtext{Considers freeway as a series of interacting streams linked by lane changes}
\item \newtext{Combination of multiple stream models}
\end{itemize}
\item \newtext{The Model}
\begin{itemize}
\item \newtext{Based on the Kinematic Wave model: \url{https://en.wikipedia.org/wiki/Kinematic_wave}}
\item \newtext{Not really sure of the maths here. More research required.}
\end{itemize}
\end{itemize}

\section{\citetitle{Kesting2007}}
\label{sec:Kesting2007}

\begin{itemize}
\item \newtext{Introduction}
\begin{itemize}
\item \newtext{Drivers want to increase their own utility}
\item \newtext{Drivers have a strategic view of lane changes -> They have a target destination that might require them to change to a specific lane}
\item \newtext{"Politeness factor" - Drivers often consider the loss of utility of other drivers. Introducing a politeness parameter varies a driver's response from altruistic to egotistical}
\item \newtext{Optimal politeness parameter -> MOBIL: Minimizing Overall Braking Induced by Lane Changes}
\item \newtext{Consider US driving rules and European driving rules (Symmetric vs Asymmetric with reversed Asymmetric for the UK)}
\end{itemize}
\item \newtext{The lane-changing model MOBIL}
\begin{itemize}
\item \newtext{Safety criterion says that after a lane change deceleration of car behind doesn't exceed a given safety limit.}
\item \newtext{Incentive to change needs to be greater than the switching threshold, which is drivers utility gain + politeness factor * follower's utility gain}
\item \newtext{Right lane bias with left lane priority for European rule roads}
\end{itemize}
\item \newtext{Application to multi-lane traffic simulations}
\begin{itemize}
\item \newtext{Intelligent Driver Model -> Guarantee's crash free driving. \url{http://www.traffic-simulation.de/}}
\item \newtext{Maximum politeness = Maximum throughput on both lanes. As density of traffic starts to get over 20/km/lane lane changes decrease as fewer suitable gaps start to appear.}
\end{itemize}
\end{itemize}

\section{\citetitle{Gipps1986}}
\label{sec:Gipps1986}
\begin{itemize}
\item \newtext{Introduction}
\begin{itemize}
\item \newtext{Modelling individual driver behaviour makes it easier to deal with bottlenecks such as road works or accidents. This behaviour is easier to simulate at a driver level. Modelling only driver behaviour instead of systems for dealing with the overall system (centralised).}
\item \newtext{Drivers need to reconcile short and long term aims.}
\item \newtext{This paper refers to how the decision to change lanes is made, as opposed to the mechanics of changing lanes.}
\end{itemize}
\item \newtext{Concerning Driver Behaviour}
\begin{itemize}
\item \newtext{Three questions:}
\begin{enumerate}
\item \newtext{Is it possible to change lanes?}
\item \newtext{Is it necessary to change lanes?}
\item \newtext{Is it desirable to change lanes?}
\end{enumerate}
\item \newtext{Assumptions include: Drivers has a goal to travel from X to Y in safety within a given time. This is translated to a number of specific quantitative objectives.}
\item \newtext{Factors influencing driver changing lanes decision}
\begin{enumerate}
\item \newtext{Physically possible and safe to change lanes}
\item \newtext{Location of permanent obstructions}
\item \newtext{The presence of transit lanes}
\item \newtext{Driver's intended turning movement}
\item \newtext{Presence of heavy vehicles}
\item \newtext{Speed}
\end{enumerate}
\end{itemize}
\item \newtext{The Model}
\begin{itemize}
\item \newtext{The model covers the entire motorway experience. Entering -> Travelling -> Exiting}
\item \newtext{Behaviour changes based on proximity to driver's exit.}
\item \newtext{Fits in with car following model. Same as that adapted in \citet{Kesting2007}}
\item \newtext{Flowchart summarising decision process as well as mathematical decision process in paper.}
\end{itemize}
\end{itemize}



\chapter{Problem Analysis}
\label{cha:Problem Analysis}

\section{Lane Merging Problems}
\label{sec:Lane Merging Problems}
Vehicles may have to merge into another lane for a number of reasons. In this paper we focus on merges made at 'critical positions' such as junctions. This analysis could later be applied to merges made at non-critical points, though centralised approaches may struggle here.

\subsection{Single-to-Single Merge}
\label{subsec:Single-to-Single Merge}
A single-to-single merge (S2S merge) describes a situation where a vehicle moves from a single lane road into another single lane road, as seen in Figure \ref{fig:S2SMerge}. In this situation we label the lane that vehicles are moving from the 'current lane' (CL), and we label the lane that vehicles move to the 'target lane' (TL). We describe the vehicles that start on the CL as 'merging vehicles' (MV) and the vehicles that start on the TL as 'target vehicles' (TV). We have our critical position where the CL and TL connect.

\begin{figure}[htb]
\includegraphics[width=\textwidth]{lane_diagrams/s2s.png}
\caption{A road with a single-to-single lane merge (S2S)}
\label{fig:S2SMerge}
\end{figure}

The main issue with an S2S merge stems from the limited options available to vehicles arriving at the critical position. Target vehicles do not have the opportunity to move laterally out of the way of merging vehicles, and vehicles on both lanes could struggle to reduce their velocity without affecting their successors.

\subsection{Single-to-Single Merge with slip-road}
\label{subsec:Single-to-Single Merge with slip-road}{}{}
\newtext{Many S2S merges are performed with an attached slip-road, as seen in Figure \ref{fig:S2SMergeExtended}.}

\begin{figure}[htb]{}
\includegraphics[width=\textwidth]{lane_diagrams/s2sExtended.png}
\caption{A road with a single-to-single lane merge and slip-lane (S2S)}
\label{fig:S2SMergeExtended}
\end{figure}

\newtext{The slip-road gives merging vehicles time to travel parallel to the target lane before merging. This makes the merge easier for both MVs and TVs as MVs don't slow down in front of TVs in order to make the turn into the TL. The effectiveness of slip-roads should change with length: the longer the slip-road, the more time MVs have to merge. This should improve the effectiveness of the merge position. We can vary the length of the slip-road to see how the performance of the merge changes.}

\subsection{Single-to-Double Merge}
\label{subsec:Single-to-Double Merge}
A single-to-double merge (S2D merge) describes a situation where a vehicle moves from a single lane road into a double lane road, as seen in Figure \ref{fig:S2DMerge}. In this situation we have two target lanes. The upper lane which directly links to the merging lane is called 'target lane 1' (TL1) and the lower lane is called 'target lane 2' (TL2). We still have only one critical position where the merging lane meets TL1.

\begin{figure}[htb]
\includegraphics[width=\textwidth]{lane_diagrams/s2d.png}
\caption{A road with a single-to-double lane merge (S2D)}
\label{fig:S2DMerge}
\end{figure}

An S2D merge provides more options for vehicles on the targets lanes at the critical position. Target vehicles now have the opportunity to move laterally to avoid merging vehicles. Two lanes also allows for more vehicles on the target lane which should give vehicles greater freedom to adjust their velocity without affecting their successors.

\subsection{Single-to-Double Merge with slip-road}
\label{subsec:Single-to-Double Merge with slip-road}

\newtext{S2D merges can also take advantage of a slip-road, as seen in Figure \ref{fig:S2DMergeExtended}}

\begin{figure}[htb]
\includegraphics[width=\textwidth]{lane_diagrams/s2dExtended.png}
\caption{A road with a single-to-double lane merge and slip-lane (S2D)}
\label{fig:S2DMergeExtended}
\end{figure}

\subsection{Double-to-Double Merge}
\label{subsec:Double-to-Double Merge}
A double-to-double merge (D2D merge) describes a situation where a vehicle moves from a double lane road into another double lane road, as seen in Figure \ref{fig:D2DMerge}. We now have two merging lanes. The upper lane, 'merging lane 1' (ML1) merges into TL1 and the lower lane, 'merging lane 2' (ML2) merges into TL2.

\begin{figure}[htb]
\includegraphics[width=\textwidth]{lane_diagrams/d2d.png}
\caption{A road with a double-to-double lane merge (D2D)}
\label{fig:D2DMerge}
\end{figure}

\newtext{With a D2D merge we now have to consider the effect of merging vehicles from ML2 driving across TL1. In addition, target vehicles can no longer laterally move out of the way of merging vehicles as they did before. Combining these factors with the wider range of options available to MVs, we can see that a D2D merge is far more complex than an S2D merge.}

\subsection{Double-to-Double Merge with slip-road}
\label{subsec:Double-to-Double Merge with slip-road}

\newtext{D2D merges can also take advantage of a slip-road, as seen in Figure \ref{fig:D2DMergeExtended}}

\begin{figure}[htb]
\includegraphics[width=\textwidth]{lane_diagrams/s2dExtended.png}
\caption{A road with a single-to-double lane merge and slip-lane (S2D)}
\label{fig:D2DMergeExtended}
\end{figure}

\newtext{D2D merges with a slip-road work differently to other slip-road schemes. In this instance vehicles on ML1 and ML2 will both merge into TL1. However, ML2 merges into TL1 as vehicles on an S2D merge would. ML1 vehicles merge into TL1 as vehicles on an S2D merge would when there is a slip-road in play.}

\subsection{Lane Obstruction Merge}
\label{subsec:Lane Obstruction Merge}
A lane obstruction merge is where a vehicle needs to change lanes to avoid an obstacle in their way, as seen in Figure \ref{fig:LaneObstruction}. It is essentially an S2S merge although the vehicle will tend to move laterally to avoid the obstacle. In this situation the critical position is the obstruction on the CL. 

\begin{figure}[htb]
\includegraphics[width=\textwidth]{lane_diagrams/lane_blocker.png}
\caption{A road with a lane obstruction}
\label{fig:LaneObstruction}
\end{figure}

The obstacle could be a broken down vehicle or some debris on the road. Because of the unexpected nature of the obstacle it may sometimes be difficult to have a centralised approach to the problem. Although, if the obstacle was a broken down vehicle, the vehicle might be able to act as the centralised system managing approaching vehicles. 

\section{Measuring Success}
\label{sec:Measuring Success}
In order to evaluate the effectiveness of solutions to the problems we need to define measurements of success. 

Solutions to the merging problems above have to satisfy the following conditions:

\begin{enumerate}
\item \textit{No collisions}
This means avoiding collisions at the critical position between merging vehicles and target vehicles, as well as avoiding collisions between vehicles on the same lane.
\item \textit{Minimise delays to both lanes}
Vehicles should not suffer large delays to travel time due to the merge. \newtext{This means measuring both average delay and maximum delay. We do not want vehicles on one lane starving (not moving) for the benefit of vehicles on the other lane.}
\item \textit{Maximise throughput}
By minimising delays and velocity loss we aim to maximise the throughput of the critical position.
\item \textit{Minimise changes in velocity}
\newtext{Though not necessary, we should aim to minimises changes in vehicle velocity, for both passenger comfort and vehicle efficiency.}
\end{enumerate}

We need to measure how well solutions meet these conditions. 

\subsection{Collisions}
\label{subsec:Collisions}
Preventing collisions is a basic safety requirement for any autonomous vehicle system. We can measure this by comparing the positions of vehicles in the system, and ensuring that there is no overlap.

\newtext{We should also consider measuring near misses. We can define a minimum spacing between vehicles, perhaps equal to the minimum braking distance of the vehicle plus an additional comfort distance. This would mimic the IDM model \citep{Treiber2000}.}

Any collisions that do happen should be reported immediately. The system should automatically be considered a failure.

\subsection{Delay}
\label{subsec:Delay}
Delay measures the effect that the critical position had on the overall journey of the vehicle. It is the primary metric considered in Dresner et al.'s 2004 paper \citep{Dresner2004} on AIM. We will measure delay in a similar manner, calculating both average delay and maximum delay.

Dresner et al. provide the following equation for measuring average delay.

\begin{equation}
\frac{1}{|C|}\sum_{v_i\in{C}}\bigl(t(i) - t_0(i)\bigr)
\end{equation}

$C$ is the set of vehicles that pass through a critical position within a set time frame. Assuming no other vehicles on the road, a vehicle $v_i$ would complete it's trip in time $t_0(i)$, otherwise $v_i$ would complete it's trip in time $t(i)$. We can represent this trip for vehicles in the simulator as the time difference between the vehicle spawning in and the vehicle being removed from the simulator.

Dresner et al. also provide the following equation for measuring maximum worst case delay:

\begin{equation}
\max_{v_i\in{C}}\bigl({t(i)} - t_0(i)\bigr)
\end{equation}

\newtext{Measuring maximum delay (and minimising it) is important, as we do not want to have a solution where some vehicles have extremely large delay times and others have very low delay times. This should help avoid a 'starvation' situation where some vehicles never get to complete their trips.}

\subsection{Throughput}
\label{subsec:Throughput}
By minimising delay we should also maximise throughput; the two are closely related. However we should also collect direct metrics.

\begin{equation}
\text{Vehicle throughput} = \frac{|C|}{t}
\end{equation}

Here $t$ is the time it took for all of the vehicles in $C$ to pass through the critical position. If we want to measure the rate at which merging vehicles and target vehicles pass through the intersection separately we can change the definition of $C$ to reflect that.

\subsection{Velocity Changes}
\label{subsec:Velocity Changes}
\newtext{We want to reduce velocity changes as much as possible, aiming especially to eliminate rapid changes. Ideally autonomous vehicles should have very smooth acceleration and braking profiles. This both increases passenger comfort and improves fuel efficiency.}

\todo{Ask Lilian for ideas on best measurements.}

Current thoughts on measurement:

Maximum decleration: How to measure? Sample each second to see changes? More frequently than that? How large a sample is too big or too small?

Similar questions for maximum acceleration.

Should we also measure net velocity change over the whole critical position?
\chapter{Design}
\label{cha:Design}

\section{S2S}{}
\label{sec:S2S}
The S2S merge is the simplest of those described in Chapter \ref{cha:Problem Analysis}. Any merging system must perform well in the S2S merge before being developed further to tackle more complex merge problems.

\subsection{Map}
\label{subsec:Map}
In order to build the map using the user's parameters we will need to calculate the relative positions of the lane entrances and exits as well as the locations of the data collection lines and spawn points.

To start with, we need to calculate the dimensions of the merging zone. The height of the merging zone will be the same as lane width of the target lane. The length can be calculated using the right-angled triangle in Figure \ref{fig:mergingZoneTriangle}. Using this triangle and some trigonometry we can calculate the length of the merge zone ($h$ in Fig. \ref{fig:mergingZoneTriangle}) using equation \ref{hSin}.

\begin{figure}[htb]
\includegraphics[width=\textwidth]{designNotes/mergingZoneTriangle.png}
\caption{A right-angled triangle used to calculate the size of the merging zone.}
\label{fig:mergingZoneTriangle}
\end{figure}

\begin{equation}\label{hSin}
h = o / \sin(\theta)
\end{equation}

We also need to know whether the horizontal width of the merge lane on the map, or it's 'base width' is longer than the target lane's lead in distance, plus the merge zone length. This will determine the width of the overall map, as if the merge lane's base length is longer then the target lane will not start with co-ordinate $x=0$ as it would if the target lane determined the width of the map.

Firstly we need to calculate the X and Y adjustments at the merge lane entrance. Because the vehicles drive in the centre of the lane and the merge lead in distance is defined by the middle line of the lane we still need to calculate how far the lane extends in the x and y directions due to it's width. To do this we can use the right-angled triangles shown in Figure \ref{fig:mergeEntranceTriangles}

\begin{figure}[htb]
\includegraphics[width=\textwidth]{designNotes/mergeEntranceTriangles.png}
\caption{Two right-angled triangles used to calculate the x and y adjustments for the merge entrance.}
\label{fig:mergeEntranceTriangles}
\end{figure}

These triangles have the same dimensions and have an interior angle of 90 - $\theta$ due to the 'alternate angle' or 'z-angle' rule. Each triangle has a hypotenuse with a length equal to half the width of the lane. 

The x-adjustment for the merge entrance is the length of the adjacent side of one of the lower triangle and the y-adjustment for the merge entrance is the length of the opposite side of the upper triangle (though both triangles do have the same dimensions). We can use equation \ref{aCos90} to calculate the X-adjustment and equation \ref{oSin90} to calculate the Y-adjustment.

\begin{equation}\label{aCos90}
a = h \cos(90 - \theta)
\end{equation}{}

\begin{equation}\label{oSin90}
o = h \sin(90 - \theta)
\end{equation}

To calculate the 'base width' of the merge lane we will also need to calculate the adjacent side of the triangle in Figure \ref{fig:baseWidthTriangle}. In this triangle the hypotenuse has a length equal to the merge lead in distance. Therefore, we can use equation \ref{aCos} to calculate the length of the adjacent side. After obtaining the length of this side we simply add the merge entrance X-adjustment and half the length of the merge zone to find the merge base width.

\begin{figure}[htb]
\includegraphics[width=\textwidth]{designNotes/baseWidthTriangle.png}
\caption{A right-angled triangle used to help calculate the base width of the merge lane, along with the X-adjustment and merge-zone length.}
\label{fig:baseWidthTriangle}
\end{figure}

\begin{equation}\label{aCos}
a = h \cos(\theta)
\end{equation}

We also need to find the point at which the merging lane's centre line crosses the target lane's centre line in the merge zone. We know the Y-coordinate for this point as it will be the same as the Y-coordinate of the target lane centre line. We also know the X-coordinate of the point at which the merge lane's centre line meets the target lane. We can use these two co-ordinates to create the triangle shown in Figure \ref{fig:toCentreTriangle}. We can then use equation \ref{aTan} to find the X-adjustment from the merge zone centre to the point where the two centre lines cross.

\begin{figure}[htb]
\includegraphics[width=\textwidth]{designNotes/toCentreTriangle.png}
\caption{A right-angled triangle used to calculate where the two centre lines meet. The centre lines are indicated in pink.}
\label{fig:toCentreTriangle}
\end{figure}

\begin{equation}\label{aTan}
a = o / \tan(\theta)
\end{equation}

\subsection{Intersection Management System}
\label{subsec:Intersection Management System}

\subsection{Decentralised Communication System}
\label{subsec:Decentralised Communication System}

\subsection{Measuring Success}
\label{subsec:Measuring Success}

\chapter{Implementation}
\label{cha:Implementation}


\section{Generalising the Codebase}
\label{sec:Generalising the Codebase}
The simulator for this project was created using the AIM simulator codebase \citep{AIMWebsite}, adapted so that new simulations can be run with the GUI without affecting the existing work. This was a project restriction imposed for research purposes. By working with the AIM simulator codebase we can determine whether it will be a good codebase to continue expanding upon for future AV projects. The project: 'A self-organising approach to autonomous vehicle car park management using a message-based protocol' \todo{Add reference}, also uses simulators built using AIM. Each simulator work alongside both the AIM and Merge simulations whilst being completely independent; removing their code would not affect the running of the other simulators. To enable this I worked closely with their project lead to generalise the codebase, breaking out useful shared features so that they can be accessed by all simulator types.



To generalise the codebase we refactored key classes into separate general and AIM specific classes. The general classes can be expanded to create other simulator specific classes, whilst the AIM specific classes maintain the functionality of the original simulator. This helps to reduce code duplication when developing new simulators.

All class diagrams were created using IntelliJ IDEA 15.0.3 internal diagram tool. Figure \ref{fig:classDiagramKey} provides a key for understanding these diagrams.

\begin{figure}[htb]
\includegraphics[width=\textwidth]{classDiagrams/classDiagramKey.png}
\caption{Key for the class diagrams in this report.}
\label{fig:classDiagramKey}
\end{figure}

\newtext{Refactoring the AIM codebase was more difficult that originally expected and many changes were made. For brevity, I have only described one of the key parts of the refactor, changing the class structure for classes in \emph{aim4.vehicle}. Appendix \ref{sec:Generalising the Codebase Appendix} provides detailed coverage of both this change, and changes made to some of the other areas of the AIM codebase.}

\todo{Describe this change}.

\section{GUI}
\label{sec:GUI}

\section{Map}
\label{sec:Map}

\section{Simulation}
\label{sec:Simulation}

\section{Merge Schemes}
\label{sec:Merge Schemes}

\section{Results Production}
\label{sec:Results Production}

\section{Testing}
\label{sec:Testing}

\subsection{Unit Testing}
\label{subsec:Unit Testing}
Unit tests were mostly used to ensure getter and setter methods worked as expected. However, some unit tests were used to verify the behaviour of classes. To do this I used Mockito \citep{MockitoWebsite} to mock the behaviour of objects used by the test class so that I could prompt the test class into producing the expected results.

\subsection{Integration Tests}
\label{subsec:Integration Tests}


\chapter{Results}
\label{cha:Results}
The simulator was designed to allow for variance in merge angle, lead in distances, speed limits, and traffic levels. This means that we can experiment to see what effect each of these variables has on the effectiveness of the Queue protocol. We can also compare the Queue protocol to the AIM protocol, using the modified version of the AIM simulator described in \ref{sec:Merge Schemes}.

\section{Experimental Procedure}
\label{sec:Experimental Procedure}
All experiments were done using pre-generated spawn schedules. In each experiment I used 20 pairs of schedules (1 schedule per lane). Schedule pairs are identical for tests with the same speed limit and traffic density (or traffic rate). Vehicles spawned for 1000 simulated seconds, and all vehicles were allowed to complete. The spawn schedules would only fail to spawn a vehicle if the spawning area was occupied by another vehicle. This can cause reduced numbers of completed vehicles if the system becomes congested enough to cause queues up to the spawning area.


\section{Comparing AIM and Queue Protocols}
\label{sec:Comparing AIM and Queue Protocols}
By using the modified AIM simulator described in \ref{sec:Merge Schemes} I obtained approximations for how well the AIM protocol handles merges.

The AIM simulator has a lead in and lead out distance for each lane of 150 metres and is limited to 90\degree merges. These settings were duplicated for the Queue merge type. All of the lanes were set to have a speed limit of 20 metres per second ($44.7\si{mph}$ or $72\si{kph}$). The traffic rate (vehicles/hour/lane) was altered to see how well the systems adjust to increasing levels of traffic.

Average delay
- AIM generally performed better
- Queue performs terribly with extremely large traffic rates 45.78 average with std dev of 18.13 vs 5.43 with 6.25. 
- Queue better for target lane at 500 and 1000. Average delay of 0.06 vs 0.37 at target lane (500). Std Dev 0.25 vs 0.79 -> Not just chance.

Plot comparing average delay of both systems

- For AIM merge performs slightly better until 1500, at which point merge starts to perform more poorly than target. 
- Merge performs more poorly throughout Queue. 

2 Plots comparing performance on each lane for AIM then Queue

Throughput
- Similar numbers of vehicles throughput until larger traffic rate, at which point AIM deals with the situation far better. At  2500 AIM deals with an extra 366 vehicles per hour.

Plot showing throughput

Reasons for performance
- AIM makes better use of space-time
- At higher traffic rates the more controlled access makes sense
- Makes a good case for further research developing an AIM based merge system.
	- Problem with current system is it hasn't been tested for shallower angles.
	- Collision detection failure is still a problem. 
	- Not a perfect representation to compare to. 

Is Queue terrible?
- It works well enough for lower traffic rates but fails abysmally at larger volumes.

\section{The Effect of the Merge Angle}
\label{sec:The Effect of the Merge Angle}
1000 Traffic Level, 40m/s, 150 metres lead in

Average Delay
- Terrible at shallow angles, hitting 74.43 at 15 degrees with standard dev of 33.19
- Performance is fairly consistent after that, though it does improve slowly as the angle hits 90 and the merge zone reaches it's shortest length.

Throughput
Very low merge throughput at shallow angles. 

Reasons
- At low angles the width of the merge zone becomes very large. At 5 degrees 45.9 metres with 4 width lane. These results are very unrealistic as such and the first 15 degrees of results are probably not particularly helpful. Shallow merges are normally dealt with using a slip road.
- Low merge throughput at shallow angles likely due to traffic backing up to the no vehicle zone 

\section{The Effect of Lead in Distances}
\label{sec:The Effect of Lead in Distances}
1000 Traffic Level, 40m/s, 45 degrees

\section{The Effect of Differing Speed Limits}
\label{sec:The Effect of Differing Speed Limits}
1000 Traffic Level, 150 metres lead in, 45 degrees.

Lilian Notes:
\begin{enumerate}
\item It will be interesting to have a graph with |C| on the x-axis and throughput on y-axis. Similarly, a 3D graph where  |Ctl| on the x-axis, |Cml| on the y-axis, and throughput on z-axis.
\end{enumerate}


\chapter{Conclusion}
\label{cha:Conclusion}
- As a "things I'd do differently" or "changes to make" it would be cool to move to a Spring implementation. It might make it easier to add different simulator types.
- Possible research : Behaviour of platoons and merging systems for them

\bibliography{bibliography}
\label{sec:Bibliography}

\begin{appendices}
\chapter{Appendix}
\label{cha:Appendix}

\section{Literature Review Further Content}
\label{sec:Literature Review Further Content}

\subsection{Gipps 1981 Equations}
\label{subsec:Gipps 1981 Equations}
Gipps' 1981 car-following model paper \citep{Gipps1981} defines two equations, which provide constraints on the speed of vehicle $n$ at time $t + \tau$. $t$ is the current time and $\tau$ is the apparent reaction time, a constant for all vehicles. The first equation defines the acceleration constraint of the vehicle. It was obtained using measurements from an instrumented car.

\begin{equation}\label{Gipps1981Accel}
v_n(t+\tau) \leqslant v_n(t) + 2.5a_n\tau\Biggl(\frac{1 - v_n(t)}{V_n}\Biggr)\Biggl(\frac{0.025 + v_n(t)}{V_n}\Biggr)^{1/2}
\end{equation}

$v_n(t)$ is the speed of vehicle $n$ at time $t$. $a_n$ is the maximum acceleration the driver of vehicle $n$ wishes to undertake. $V_n$ is the target speed for vehicle $n$. The equation shows that the driver accelerates until close to their target speed. Then, they reduce their acceleration until it reaches zero. At this point the vehicle should be travelling at it's target speed.

The second constraint is the braking profile of the vehicle. This is given as

\begin{equation}\label{Gipps1981Brake}
\begin{split}
&v_n(t+\tau) \leqslant \\
&b_n\tau + \sqrt{\Biggl(b_n^2\tau^2 - b_n\biggl(2\Bigl[x_{n-1}(t) - s_{n-1} - x_n(t)\Bigr] - v_n(t)\tau - \frac{v_{n-1}(t)^2}{\hat{b}}\biggr)\Biggr)}
\end{split}
\end{equation}

$b_n$ is the most severe braking the driver of vehicle $n$ wishes to undertake. It is always a negative value, and should be considered negative acceleration. $\hat{b}$ is the driver of vehicle $n$'s best guess at $b_{n-1}$ where $n-1$ is $n$'s predecessor. $x_n(t)$ is the location of the front of vehicle $n$ at time $t$. $s_n$ is the effective size of vehicle $n$. This is equal to the physical length of $n$, plus a margin $n$'s successor is not willing to enter, even when $n$ is at rest.

Therefore, at time $t + \tau$, assuming the driver travels as fast as is safe, and within the limitations of the vehicle, their speed is given by the minimum of these two equations.

\begin{equation}
v_n(t) = \min{(\eqref{Gipps1981Accel},\eqref{Gipps1981Brake})}
\end{equation}

\subsection{The Intelligent Driver Model}
\label{subsec:The Intelligent Driver Model}
In 2000 Treiber et al. suggested the 'Intelligent Driver Model' (IDM) \citep{Treiber2000}. In the IDM, the acceleration of vehicle $\alpha$, $\dot{v_\alpha}$, is defined using a continuous function of its velocity, $v_\alpha$; the distance to the rear of its predecessor, $s_\alpha$; and the velocity difference of $\alpha$ and it's predecessor, also known as the approaching rate $\Delta v_\alpha$. The vehicle interactions are solely based on $\alpha$'s relative acceleration to its predecessor. The model only provides position information for a vehicle in relation to its predecessor, and it does not provide its velocity at a given time, as Gipps' model does. 

The IDM is broken into two components. The first describes the behaviour of a vehicle on a free road.

\begin{equation}
\dot{v_\alpha} = a^{(\alpha)}\Biggl[1 - \biggl(\frac{v_\alpha}{v_0^{(\alpha)}}\biggr)^\delta\Biggr]
\end{equation}

Here $a^{(\alpha)}$ is the maximum acceleration of vehicle $\alpha$ and $v_0^{\alpha}$ is the desired velocity of $\alpha$. $\delta$ is the acceleration exponent, which is typically 4. 

The second component describes the behaviour of a vehicle as it approaches its predecessor. 

\begin{equation}
\dot{v_\alpha} = - a^{(\alpha)}\biggl(\frac{s^*}{s_\alpha}\biggr)^2
\end{equation}

As the gap, $s_\alpha$, between $\alpha$ and it's predecessor, gets closer to the desired minimum gap $s^*$, $\alpha$ decelerates.

Interpolating the two components gives us the IDM. 

\begin{equation}
\dot{v_\alpha} = a^{\alpha}\Biggl[1 - \biggl(\frac{v_\alpha}{v_0^\alpha}\biggr)^\delta - \biggl(\frac{s^*(v_\alpha,\Delta v_\alpha)}{s_\alpha}\biggr)^2\Biggr]
\end{equation}

The desired minimum gap in the IDM varies dynamically with velocity and approaching rate. It is given by the following function.

\begin{equation}\label{IDMSpacingFunction}
s^*(v,\Delta v) = s_0^{(\alpha)} + s_1^{(\alpha)}\sqrt{\frac{v}{v_0^{(\alpha)}}} + T^\alpha v + \frac{v\Delta v}{2\sqrt{a^{(\alpha)}b^{(\alpha)}}}
\end{equation}

The equation takes the bumper-to-bumper space $s_0^{(\alpha)}$, also known as the minimum jam distance, and adds the comfortable jam distance $s_1^{(\alpha)}$. The bumper-to-bumper space is the minimum gap between $\alpha$ and its predecessor in stationary traffic. The comfortable jam distance is an extra distance added on for comfort, and to allow for a slower driver reaction time. In the paper, this value is set to $0$. We can also consider it negligible for autonomous vehicles. $T$ is the safe time headway; it represents the time required for the vehicle to safely come to a stop. Finally $b^{(\alpha)}$ is the desired deceleration for $\alpha$.

\section{S2S Map Calculations}
\label{sec:S2SMapCalculations}
To start with, we need to calculate the dimensions of the merging zone. The height of the merging zone will be the same as lane width of the target lane. The length can be calculated using the right-angled triangle in Figure \ref{fig:mergingZoneTriangle}. Using this triangle and some trigonometry we can calculate the length of the merge zone ($\text{mergingZoneLength}$ in Fig. \ref{fig:mergingZoneTriangle}) using equation \ref{hSin}.

\begin{figure}[htb]
\centering
\includegraphics[width=10cm]{appendix/mergingZoneTriangle.png}
\caption{A right-angled triangle used to calculate the size of the merging zone.}
\label{fig:mergingZoneTriangle}
\end{figure}

\begin{equation}\label{hSin}
mergeZoneLength = \frac{laneWidth}{\sin(\theta)}
\end{equation}

We also need to know whether the horizontal width of the merge lane on the map, or it's 'base width' is longer than the target lane's lead in distance, plus the merge zone length. This will determine the width of the overall map, as if the merge lane's base length is longer then the target lane will not start with co-ordinate $x=0$ as it would if the target lane determined the width of the map.

Firstly we need to calculate the X and Y adjustments at the merge lane entrance. Because the vehicles drive in the centre of the lane and the merge lead in distance is defined by the middle line of the lane we still need to calculate how far the lane extends in the x and y directions due to it's width. To do this we can use the right-angled triangles shown in Figure \ref{fig:mergeEntranceTriangles}

\begin{figure}[htb]
\centering
\includegraphics[width=10cm]{appendix/mergeEntranceTriangles.png}
\caption{Two right-angled triangles used to calculate the x and y adjustments for the merge entrance.}
\label{fig:mergeEntranceTriangles}
\end{figure}

These triangles have the same dimensions and have an interior angle of 90 - $\theta$ due to the 'alternate angle' or 'z-angle' rule. Each triangle has a hypotenuse with a length equal to half the width of the lane. 

The X-adjustment for the merge entrance is the length of the adjacent side of one of the lower triangle and the Y-adjustment for the merge entrance is the length of the opposite side of the upper triangle (though both triangles do have the same dimensions). We can use equation \ref{aCos90} to calculate the X-adjustment and equation \ref{oSin90} to calculate the Y-adjustment.

\begin{equation}\label{aCos90}
\text{x-adjust} = \frac{laneWidth}{2} \cos(90 - \theta)
\end{equation}

\begin{equation}\label{oSin90}
\text{y-adjust} = \frac{laneWidth}{2} \sin(90 - \theta)
\end{equation}

To calculate the 'base width' of the merge lane we will also need to calculate the adjacent side of the triangle in Figure \ref{fig:baseWidthTriangle}. In this triangle the hypotenuse has a length equal to the merge lead in distance. Therefore, we can use equation \ref{aCos} to calculate the length of the adjacent side. After obtaining the length of this side we simply add the merge entrance X-adjustment and half the length of the merge zone to find the merge base width.

\begin{figure}[htb]
\centering
\includegraphics[width=10cm]{appendix/baseWidthTriangle.png}
\caption{A right-angled triangle used to help calculate the base width of the merge lane, along with the X-adjustment and merge-zone length.}
\label{fig:baseWidthTriangle}
\end{figure}

\begin{equation}\label{aCos}
mergingLaneCentreLineBase = mergeLeadInDistance \cos(\theta)
\end{equation}

We also need to find the point at which the merging lane's centre line crosses the target lane's centre line in the merge zone. We know the Y-coordinate for this point as it will be the same as the Y-coordinate of the target lane centre line. We also know the X-coordinate of the point at which the merge lane's centre line meets the target lane. We can use these two co-ordinates to create the triangle shown in Figure \ref{fig:toCentreTriangle}. We can then use equation \ref{aTan} to find the X-adjustment from the merge zone centre to the point where the two centre lines cross.

\begin{figure}[htb]
\centering
\includegraphics[width=10cm]{appendix/toCentreTriangle.png}
\caption{A right-angled triangle used to calculate where the two centre lines meet. The centre lines are indicated in pink.}
\label{fig:toCentreTriangle}
\end{figure}

\begin{equation}\label{aTan}
toCentreDistance = frac{laneWidth}{2 \tan(\theta)}
\end{equation}

\section{Generalising the Codebase}
\label{sec:Generalising the Codebase Appendix}

\subsection{aim4.driver}
\label{subsec:aim4.driver}
\emph{aim4.driver} controls how a vehicle behaves on the map. In the original simulator the drivers were built to deal with 4-way intersections, with general functionality tied into the same class. You can see how this was done in Figure \ref{fig:driverBefore}.

\begin{figure}[htb]
\includegraphics[width=\textwidth]{classDiagrams/driverBefore.png}
\caption{The original class structure for \emph{Driver}.}
\label{fig:driverBefore}
\end{figure}

The first major change was renaming \emph{DriverSimView}, \emph{AutoDriverPilotView}, and \emph{AutoDriverCoordinatorView} to end in \emph{Model} instead of \emph{View}. These interfaces are used to limit the methods that other classes can access in Driver and AutoDriver, thus changing their 'view' of that class. We felt that \emph{View} could cause confusion with the GUI elements of the simulator; we instead chose to refer to these interfaces as \emph{Models}, because the accessors are effectively given a model of Driver and AutoDriver (beyond which they care very little) that they can use to access methods.

The next change was separating out all of the AIM specific code into its own classes and interfaces. You can see how this was done in Figure \ref{fig:driverAfter} with \emph{AIMDriverSimModel} and \emph{AIMDriver}. The merge specific code found in \emph{MergeDriverSimModel}, \emph{MergeDriver} and \emph{MergeAutoDriver} is structured in a very similar manner to its AIM counterpart, taking advantage of the generalised code.

\begin{sidewaysfigure}[p]
\includegraphics[width=\textwidth]{classDiagrams/driverAfter.png}
\caption{The new class structure for \emph{Driver}.}
\label{fig:driverAfter}
\end{sidewaysfigure}

As a consequence of breaking out the code like this, a number of additional changes had to be made. Driver was changed into an interface and a new class \emph{BasicDriver}. \emph{Driver} is simply used as an interface for accessing Drivers in non-simulation contexts (such as \emph{BasicVehicle}). \emph{BasicDriver} contains the generalised functionality all \emph{Driver} objects should need, with AIM specific activities moved to \emph{AIMDriver}. Extending from \emph{Driver} is the \emph{AutoDriver} interface, which adds no new methods but is instead used to categorise autonomous drivers. \emph{AIMAutoDriver} contains almost exactly the same code as the original \emph{AutoDriver} class.

\FloatBarrier
\subsection{aim4.gui}
\label{subsec:aim4.gui}
\emph{aim4.gui} controls the GUI for the simulator. We had to adjust this to allow for non-AIM simulations to be run. We chose to use tabs to allow users to switch between simulators (these are greyed out when a simulation is running). To make adding new tabs and simulation screens easier we had to refactor \emph{Viewer} into smaller, separate components. You can see the structural changes in Figures \ref{fig:originalAIMSetupLabeled} and \ref{fig:newAIMSetupLabeled}.

In the original simulator \emph{Viewer} displays the simulator set-up controls, \emph{SimSetupPanel}, and the simulation viewer \emph{Canvas} inside \emph{mainPanel}. \emph{mainPanel} is a \emph{JPanel} with a \emph{CardLayout} allowing the panel to switch between displaying the set-up controls and the viewer. In the new simulator we replaced \emph{mainPanel} with \emph{tabbedPane}, a \emph{JTabbedPane} object that allows users to switch between the different simulators using tabs. Each tab displays a \emph{SimViewer}, which behaves in a similar way to \emph{mainPanel} allowing users to switch between the set-up screen and the simulation screen using \emph{CardLayout}. Each simulator will have their own SimViewer type, as shown in Figure \ref{fig:simViewer}.

\begin{figure}[htb]
\centering
\includegraphics[width=8cm]{classDiagrams/simViewer.png}
\caption{The class diagram for \emph{SimViewer}.}
\label{fig:simViewer}
\end{figure}

We didn't want to force new simulators to use a full representation of vehicles on screen, as \emph{Canvas} does. To avoid this we created a new interface \emph{SimScreen} which \emph{SimViewer} uses to describe it's viewer card. Any class implementing \emph{SimScreen} can be used as the viewer for a simulation. Figure \ref{fig:simScreen} shows how \emph{MergeStatScreen} and \emph{Canvas} using \emph{SimScreen}.

\begin{figure}[htb]
\centering
\includegraphics[width=8cm]{classDiagrams/simScreen.png}
\caption{The class diagram for \emph{SimScreen}.}
\label{fig:simScreen}
\end{figure}

We also generalised the \emph{SimSetupPanel} class to allow \emph{SimViewer} to display non-AIM set-up controls. Figure \ref{fig:simSetupPanel} shows the new class structure for \emph{SimSetupPanel}.

\begin{figure}[htb]
\centering
\includegraphics[width=8cm]{classDiagrams/simSetupPanel.png}
\caption{The class diagram for \emph{SimSetupPanel}.}
\label{fig:simSetupPanel}
\end{figure}

We also made a small adjustment to the behaviour of the reset option in the menu. Now the simulator must be paused in order for the reset button to be active. We did this because resetting the simulator without pausing was creating \emph{NullPointerException}s.

\begin{figure}[p]
\centerline{
\includegraphics[width=15cm]{screenshots/originalAIMSetupLabeled.png}
}
\caption{Panel layout in the original simulator.}
\label{fig:originalAIMSetupLabeled}
\end{figure}

\begin{figure}[p]
\centerline{
\includegraphics[width=15cm]{screenshots/newAIMSetupLabeled.png}
}
\caption{Panel layout in the new simulator.}
\label{fig:newAIMSetupLabeled}
\end{figure}

\FloatBarrier
\subsection{aim4.map}
\label{subsec:aim4.map}
\emph{aim4.map} is used to describe the environment vehicles are required to navigate. They also spawn vehicles that then drive through the map. Figures \ref{fig:mapBefore} and \ref{fig:mapAfter} show the original and new class structure for \emph{aim4.map}.

\begin{figure}[htb]
\centering
\includegraphics[height=6cm]{classDiagrams/mapBefore.png}
\caption{The original class structure for \emph{BasicMap} and \emph{SpawnPoint}.}
\label{fig:mapBefore}
\end{figure}

\begin{figure}[htb]
\centering
\includegraphics[width=8cm]{classDiagrams/mapAfter.png}
\caption{The new class structure for \emph{BasicMap} and \emph{SpawnPoint}.}
\label{fig:mapAfter}
\end{figure}

The changes made to \emph{aim4.map} were relatively straight-forward. The AIM specific features in \emph{BasicMap} were extracted out in \emph{BasicIntersectionMap} and \emph{GridMap} was renamed to \emph{GridIntersectionMap} and now inherits from the new interface. This allows for new map types, such as \emph{MergeMap} to implement a map type without AIM features.

\emph{SpawnPoint} was also broken out into general and AIM specific features. This had to be done because \emph{SpawnPoint} used to create \emph{SpawnSpec} objects with \emph{destination} fields. \emph{destination} is an AIM specific field relating to the intersection exit a vehicle plans to reach. By extracting this out new map types can spawn vehicles with \emph{SpawnSpec} instances specific to their map type.

\FloatBarrier
\subsection{aim4.sim}
\label{subsec:aim4.sim}
\emph{aim4.sim} contains the code responsible for constructing and running simulations. The original code was very focussed on AIM simulations and so we had to break the interfaces to allow for different types of simulators. 

\emph{Simulator} is an interface that new simulators need to implement. We decided to extract out some of the AIM specific features into \emph{AIMSimulator}. We also added an override to \emph{getMap()}, forcing AIM simulators to use \emph{BasicIntersectionMap} maps. The class structure changes can be seen in Figures \ref{fig:simulatorBefore} and \ref{fig:simulatorAfter}.

\begin{figure}[htb]
\centering
\includegraphics[width=8cm]{classDiagrams/simulatorBefore.png}
\caption{The original class structure for \emph{Simulator}.}
\label{fig:simulatorBefore}
\end{figure}

\begin{figure}[htb]
\centering
\includegraphics[width=12cm]{classDiagrams/simulatorAfter.png}
\caption{The new class structure for \emph{Simulator}.}
\label{fig:simulatorAfter}
\end{figure}

\emph{SimSetup} was also modified to separate AIM specific set-up options and simulator creation code from other simulators. Figures \ref{fig:simSetupBefore} and \ref{fig:simSetupAfter} show how these classes were altered.

\begin{figure}[htb]
\centering
\includegraphics[height=6cm]{classDiagrams/simSetupBefore.png}
\caption{The original class structure for \emph{SimSetup}.}
\label{fig:simSetupBefore}
\end{figure}

\begin{figure}[htb]
\centering
\includegraphics[width=10cm]{classDiagrams/simSetupAfter.png}
\caption{The new class structure for \emph{SimSetup}.}
\label{fig:simSetupAfter}
\end{figure}

\FloatBarrier
\subsection{aim4.vehicle}
\label{subsec:aim4.vehicle}
\emph{aim4.vehicle} controls the different vehicles used during simulations. Vehicles are used by both \emph{Driver} and \emph{Simulator} instances. To allow them to do that the original simulator code used \emph{View} interfaces similar to those in \ref{subsec:aim4.driver}. Figure \ref{fig:vehicleBefore} shows how these interfaces link together. Extracting AIM behaviour was quite difficult because of how interconnected these interfaces were. The solution we came up with was to create AIM specific interfaces and link them together in a similar manner, inheriting from the generic ones if possible. Figure \ref{fig:vehicleAfter} shows how the new structure links together.

\begin{figure}[htb]
\centering
\includegraphics[width=10cm]{classDiagrams/vehicleBefore.png}
\caption{The original class structure for \emph{aim4.vehicle}.}
\label{fig:vehicleBefore}
\end{figure}

\begin{sidewaysfigure}[p]
\includegraphics[width=\textwidth]{classDiagrams/vehicleAfter.png}
\caption{The new class structure for \emph{aim4.vehicle}.}
\label{fig:vehicleAfter}
\end{sidewaysfigure}

The first change made to \emph{aim4.vehicle} was to rename all of the files ending in \emph{View} to end in \emph{Model} instead. This matches the changes made to \emph{aim4.driver}.

\emph{AIMVehicleSimModel} and \emph{AIMAutoVehicleDriverModel} are at the top of the AIM interface tree. They both extend their generic counterparts. \emph{AIMAutoVehicleSimModel} extends these two interfaces along with \emph{AutoVehicleSimModel}. This matches up to the original inheritance structure. Any future vehicles will need to create their own version of these interfaces, as seen in \emph{MergeVehicleSimModel}, \emph{MergeAutoVehicleDriverModel} and \emph{MergeAutoVehicleSimModel}. 

In terms of classes we made \emph{BasicAutoVehicle} abstract and extracted out AIM specific behaviour to \emph{AIMBasicAutoVehicle}. \emph{BasicAutoVehicle} had to be abstract because we wanted to force \emph{getDriver()} to be overridden in subclasses to retrieve the simulator specific \emph{AutoDriver} for that vehicle (for example \emph{AIMAutoDriver} in AIM simulators). 

\FloatBarrier
\section{Maps}
\label{sec:Maps}
All maps testing Merge functionality implement \emph{BasicMap}. I created a generalised implementation called \emph{MergeMap} which satisfies the basic functionality of BasicMap as well as some protected accessors. All maps used during simulations extend \emph{MergeMap}.

\subsection{MergeMapUtil}
\label{sec:MergeMapUtil}
Similar to AIM's \emph{GridMapUtil}, \emph{MergeMapUtil} provides useful functions to \emph{MergeMap}, including \emph{SpawnPoint} \emph{VehicleSpec} generators.

\end{appendices}

\end{document}
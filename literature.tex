\chapter{Literature Review}
\label{cha:Literature Review}

\section{\citetitle{Dresner2004}}
\label{sec:Dresner2004}

\begin{itemize}
\item \newtext{Introduction}
\begin{itemize}
\item \newtext{Motivated by traffic light problems -> My work is motivated by lane change problems evidenced in \citet{Laval2006}}
\item \newtext{Set in a world of fully autonomous vehicles.}
\item \newtext{Overpass is optimal solution, what is optimal for my work?}
\end{itemize}
\item \newtext{The Model}
\begin{itemize}
\item \newtext{Using a simplified model of real-world intersection traffic -> No turning, roughly same speed. Worth considering my model to start with. How can that model be adapted?}
\item \newtext{How do we measure success? 1) Safety is critical. No collisions allowed! 2) Efficiency}
\item \newtext{Throughput: How much traffic can be handled. Difficult to measure, qualitative claims only made.}
\item \newtext{Delay: Effect on overall journey of the vehicle. No vehicle's travel time sacrificed for another dramatically. Consider both average delay and maximum delay!}
\end{itemize}
\item \newtext{Overpass, Traffic Light Theory}
\begin{itemize}
\item \newtext{Simplifications due to car interactions and acceleration to calculate lower bound traffic light delays.}
\end{itemize}
\item \newtext{The Simulator}
\begin{itemize}
\item \newtext{Useful sizing stats in this section.}
\item \newtext{Spawning characteristics, driver properties, three actions the driver can take, relevant decision logic.}
\item \newtext{Testing simulator with no big changes. Constructing current system}
\end{itemize}
\begin{itemize}
\item \newtext{Simplification from earlier no longer applies to light model.}
\item \newtext{"Call ahead" system -> Could be applied to a centralised model for lane changing.}
\item \newtext{Intersection divided into reservation tiles -> could be applied to lane changing too.}
\end{itemize}
\item \newtext{Empirical Results}
\begin{itemize}
\item \newtext{Measuring overloaded systems vs light traffic}
\item \newtext{Increasing granularity tests -> At least as high as the number of lanes}
\end{itemize}
\end{itemize}

\section{\citetitle{Laval2006}}
\label{sec:Laval2006}

\begin{itemize}
\item \newtext{Introduction}
\begin{itemize}
\item \newtext{Attempt to create a qualitative understanding of lane changing impacts on traffic flow.}
\item \newtext{Lane change triggers disruption -> triggers other changes}
\item \newtext{Considers freeway as a series of interacting streams linked by lane changes}
\item \newtext{Combination of multiple stream models}
\end{itemize}
\item \newtext{The Model}
\begin{itemize}
\item \newtext{Based on the Kinematic Wave model: \url{https://en.wikipedia.org/wiki/Kinematic_wave}}
\item \newtext{Not really sure of the maths here. More research required.}
\end{itemize}
\end{itemize}

\section{\citetitle{Kesting2007}}
\label{sec:Kesting2007}

\begin{itemize}
\item \newtext{Introduction}
\begin{itemize}
\item \newtext{Drivers want to increase their own utility}
\item \newtext{Drivers have a strategic view of lane changes -> They have a target destination that might require them to change to a specific lane}
\item \newtext{"Politeness factor" - Drivers often consider the loss of utility of other drivers. Introducing a politeness parameter varies a driver's response from altruistic to egotistical}
\item \newtext{Optimal politeness parameter -> MOBIL: Minimizing Overall Braking Induced by Lane Changes}
\item \newtext{Consider US driving rules and European driving rules (Symmetric vs Asymmetric with reversed Asymmetric for the UK)}
\end{itemize}
\item \newtext{The lane-changing model MOBIL}
\begin{itemize}
\item \newtext{Safety criterion says that after a lane change deceleration of car behind doesn't exceed a given safety limit.}
\item \newtext{Incentive to change needs to be greater than the switching threshold, which is drivers utility gain + politeness factor * follower's utility gain}
\item \newtext{Right lane bias with left lane priority for European rule roads}
\end{itemize}
\item \newtext{Application to multi-lane traffic simulations}
\begin{itemize}
\item \newtext{Intelligent Driver Model -> Guarantee's crash free driving. \url{http://www.traffic-simulation.de/}}
\item \newtext{Maximum politeness = Maximum throughput on both lanes. As density of traffic starts to get over 20/km/lane lane changes decrease as fewer suitable gaps start to appear.}
\end{itemize}
\end{itemize}

\section{\citetitle{Gipps1986}}
\label{sec:Gipps1986}
\begin{itemize}
\item \newtext{Introduction}
\begin{itemize}
\item \newtext{Modelling individual driver behaviour makes it easier to deal with bottlenecks such as road works or accidents. This behaviour is easier to simulate at a driver level. Modelling only driver behaviour instead of systems for dealing with the overall system (centralised).}
\item \newtext{Drivers need to reconcile short and long term aims.}
\item \newtext{This paper refers to how the decision to change lanes is made, as opposed to the mechanics of changing lanes.}
\end{itemize}
\item \newtext{Concerning Driver Behaviour}
\begin{itemize}
\item \newtext{Three questions:}
\begin{enumerate}
\item \newtext{Is it possible to change lanes?}
\item \newtext{Is it necessary to change lanes?}
\item \newtext{Is it desirable to change lanes?}
\end{enumerate}
\item \newtext{Assumptions include: Drivers has a goal to travel from X to Y in safety within a given time. This is translated to a number of specific quantitative objectives.}
\item \newtext{Factors influencing driver changing lanes decision}
\begin{enumerate}
\item \newtext{Physically possible and safe to change lanes}
\item \newtext{Location of permanent obstructions}
\item \newtext{The presence of transit lanes}
\item \newtext{Driver's intended turning movement}
\item \newtext{Presence of heavy vehicles}
\item \newtext{Speed}
\end{enumerate}
\end{itemize}
\item \newtext{The Model}
\begin{itemize}
\item \newtext{The model covers the entire motorway experience. Entering -> Travelling -> Exiting}
\item \newtext{Behaviour changes based on proximity to driver's exit.}
\item \newtext{Fits in with car following model. Same as that adapted in \citet{Kesting2007}}
\item \newtext{Flowchart summarising decision process as well as mathematical decision process in paper.}
\end{itemize}
\end{itemize}



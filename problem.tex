\chapter{Problem Analysis}
\label{cha:Problem Analysis}
\newtext{All new}

\section{Lane Merging Problems}
\label{sec:Lane Merging Problems}
Vehicles may have to merge into another lane for a number of reasons. In this paper we focus on merges made at 'critical positions' such as junctions. This analysis could later be applied to merges made at non-critical points, though centralised approaches may struggle here.

\subsection{Single-to-Single Merging}
\label{subsec:Single-to-Single Merging}
A single-to-single merge (S2S merge) describes a situation where a vehicle moves from a single lane road into another single lane road, as seen in Figure \ref{fig:S2SMerge}. In this situation we label the lane that vehicles are moving from the 'current lane', and we label the lane that vehicles move to the 'target lane'. We describe the vehicles that start on the current lane as merging vehicles and the vehicles that start on the target lane as target vehicles.

\begin{figure}[htb]
\includegraphics[width=\textwidth]{ImageMissing.png}
\caption{A road with a single-to-single lane merge (S2S)}
\label{fig:S2SMerge}
\end{figure}

An S2S merge limits the options available to vehicles arriving at the critical position. Target vehicles do not have the opportunity to move laterally out of the way of merging vehicles and vehicles on both lanes could struggle to reduce their velocity without affecting their successors.

\subsection{Single-to-Double Merging}
\label{subsec:Single-to-Double Merging}

\subsection{Double-to-Double Merging}
\label{subsec:Double-to-Double Merging}

\section{Measuring Success}
\label{sec:Measuring Success}